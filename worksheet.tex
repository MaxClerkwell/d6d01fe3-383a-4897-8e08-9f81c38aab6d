\documentclass{dcbl/challenge}

\setdoctitle{Redirecting Streams}
\setdocauthor{Stephan Bökelmann}
\setdocemail{sboekelmann@ep1.rub.de}
\setdocinstitute{AG Physik der Hadronen und Kerne}


\begin{document}

One of the commonly used features when it comes to programming is writing a series of characters to the terminal.
A concept closely related to that idea of printing a series of characters is called a stream.
At this point of our journey, we are not trying to define the exact meaning of the word stream, but rather just the idea of reading and writing characters.
What a program is writing, we now call an output stream, a series of characters that a program reads an input stream.
In this exercise we want to investigate, how we can redirect outputs of a program and have it send its stream somewhere else, than our terminal-buffer.

\section*{Exercises}
\begin{aufgabe}
    To work with streams, we first of all need a file with a bunch of content. 
    We could either use a poem, a list of numbers, or a random text, which is stored in our file-system.
    Feel free to use any of those, or choose to use \texttt{/dev/urandom} as an example.
    When you read from this particular file, the operating system will generate a pseudo-random stream of characters. 
    Opening it with \texttt{cat /dev/urandom} will definitly spam the terminal with random characters, since it is an infinite stream.
    Reading from this file can easily crash your terminal, since random numbers can be generated way faster than the rest of the processes can handle.
    In order for us, not to crash the terminal, we need to redirect this stream somewhere else.
    To do this, we can use the \texttt{>} operator.
    Navigate into your \texttt{$\sim$}-directory and use the following command, to read from \texttt{/dev/urandom} and write the stream directly into a file called \texttt{random.txt}: \texttt{cat /dev/urandom > random.txt}.
    Stop the process any time by pressing \texttt{Ctrl+C}.
    Repeat the process with a new file calles \texttt{random2.txt} and redirect the stream to that file, but vary the time it takes you to stop the process.
    Use the command \texttt{ls -la} to list all files in your \texttt{$\sim$}-directory and compare the size of the two files.
    Write down, what you have learned from this.
\end{aufgabe}

\begin{aufgabe}
    No matter how fast you stopped the previous process, the files will hold a significant amount of data. 
    Let's find out how many lines and how many words our files contain by using a different program called \texttt{wc}.
    We used the \texttt{>}-operator earlier. 
    This is a special operator which tells our shell, that we'd like to have our output be written to a file.
    In order to have the output redirected to another program we have to use the \texttt{|}-operator.
    Let's run the command \texttt{cat random.txt | wc} and observe the output.
    \texttt{wc} will count the number of lines and words in the file and write it to the terminal.
    We can redirect this output again to a file called \texttt{result1.txt} by writing \texttt{cat random.txt | wc > result1.txt}.
\end{aufgabe}

\begin{aufgabe}
    We could do the same thing for the second file.
    But for more than two files, this could be a bit cumbersome.
    Let's do this in a different way by using the \texttt{>>}-operator.
    Instead of redirecting the output to another file, we can use the \texttt{>>}-operator to append the output to an existing file.
    Let's run \texttt{cat random2.txt | wc >> result1.txt} and observe the output.
\end{aufgabe}

\begin{aufgabe}
    In the case of many files, it could be hard to read the output.
    Let's make this more verbatim by adding some comments to the file and running it all again in one line. Run\\
    \begin{itemize}
        \item \texttt{\{ echo -n "File \#1: "; cat random.txt | wc ; echo -n "File \#2: "; cat random2.txt | wc; \} > result.txt}
    \end{itemize}
    By using the \texttt{\{ and \}}-operator, we can group multiple commands in one line.
\end{aufgabe}

\begin{aufgabe}
    In order to search for a specific string inside of a stream, we can use the \texttt{grep} command.
    Let us download the \textbf{RFC 2828}, which is the \textit{Internet Security Glossary} and search in this file for all occurences of \texttt{security}.
    Run: \\
    \begin{itemize}
        \item \texttt{wget https://www.ietf.org/rfc/rfc2828.txt -O - | grep 'security'}
    \end{itemize}
    \texttt{wget} loads the file from its URI and \texttt{-O -} redirects the output to the standard output stream.
    The standard output is then redirected to the program \texttt{grep} and all occurences of \texttt{security} will be marked in the terminal.
    Try using \texttt{grep} to search in a list of files for a specific name.
\end{aufgabe}

\begin{aufgabe}
    Another tool, commonly used in this concatenation of stream processing is \texttt{sed}, or \textit{stream editor}.
    It is a configurable program that allows us to manipulate the content of a stream.
    Let's try and replace every occurence of \texttt{security} with \texttt{AHHH} in the download of \textbf{RFC 2828}.
    \begin{itemize}
        \item \texttt{wget https://www.ietf.org/rfc/rfc2828.txt -O - | sed 's/security/AHHH/g' | grep 'AHHH'}
    \end{itemize}
\end{aufgabe}

\section*{Annotations}
\begin{enumerate}
    \item Redhat Tutorial on redirects: \url{https://www.redhat.com/sysadmin/linux-shell-redirection-pipelining}
\end{enumerate}

\end{document}
